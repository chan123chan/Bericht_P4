% !TEX root = Fachbericht.tex

%Verwendet die hier aufgeführten Beispiele damit keine Probleme entstehen.

%section  = 3.Hardware
%subsection = 3.1 Hardware
%subsubsection = 3.1.1 Hardware
%wird ein Stern angefügt, erscheint keine Nummerierung und die Auflistung im Inhaltsverzeichnis wird unterdrückt



\subsection{Beispielabschnitt} \label{sec:beispielabschnitt}

%Beispiel für das Einfügen eines Bildes, inklusive der Beschriftungen und falls nötig Quellenverweis: 
%Achtung: \caption[Steht im Abbildungsverzeichnis]{unter dem Bild}
%\label{fig: als Referenz für spätere Verweise, gewählter Name erscheint}
%Vorsicht: das Bild muss exakt die gleiche Beschriftung im Dateinamen haben, wie im Pfad Data/Spannungsnetzebenen, sonst funktioniert es nicht.

\begin{figure}[htbp]
\centering
\includegraphics[width=1\textwidth]{Data/Spannungsnetzebenen}
\caption[Netzebenen\cite{eetgl_skript}]{Netzebenen}
\label{fig:Netzebenen}
\end{figure} 

%2 Beispiele für Formeln:

So kann aus den Dokumenten \cite{Niklaus_Skript}, \cite{ant_skript}, \cite{dt1_skript}, \cite{eetgl_skript}, \cite{mc1_skript} und \cite{INA_128} entnommen werden, dass die folgenden Beziehungen korrekt sind:

\begin{equation}
\centering
u(t) = \^{U} \cdot \cos (\omega t + \varphi u)
\label{eq:Spannungsfunktion}
\end{equation}

\begin{equation}
\centering
U = U_{eff} = \sqrt{\dfrac{1}{T}\cdot \int_0^T u^2(t) dt}
\label{eq:Effektivwert}
\end{equation}

Anschliessend kann im Text der Verweis gemacht werden. Mit der Gleichung \ref{eq:Spannungsfunktion} und \ref{eq:Effektivwert} wird die Abbildung \ref{fig:Netzebenen} selbsterklärend. Ebenfalls kann auf das Kapitel \ref{sec:grundlagen} verwiesen werden. Noch zwei Beispiele mit Tabelle \ref{tab:BelasteteMessung} und \ref{tab:SpannungsmessungTabelle}. Es gibt auch einen Tabellengenerator für LATEX, sowie das Programm JabRef, mit welchem bereits der entsprechende Code für das bibtex-file quelle erstellt.

\begin{table}[htbp]
\centering
\begin{tabular}{lllllllllll}
\multicolumn{3}{c}{\textbf{Wattmeter}}                       & \textbf{} & \multicolumn{3}{c}{\textbf{Projekt}}                         & \textbf{} & \multicolumn{3}{c}{\textbf{Abweichung}}                         \\
\textbf{P {[}W{]}} & \textbf{U {[}V{]}} & \textbf{I {[}A{]}} & \textbf{} & \textbf{P {[}W{]}} & \textbf{U {[}V{]}} & \textbf{I {[}A{]}} & \textbf{} & \textbf{P {[}\%{]}} & \textbf{U {[}\%{]}} & \textbf{I {[}\%{]}} \\ \cline{1-3} \cline{5-7} \cline{9-11} 
5.48             & 230.7             & 0.024              &           & 5.00             & 230.58             & 0.0328              &           & 8.7                & 0.05                & 36.6                \\
8.63              & 230.6             & 0.0377              &           & 8.00              & 230.34             & 0.0376              &           & 7.3                & 0.11                & 0.27                \\
73.72              & 235.90             & 0.312              &           & 72.00              & 235.04             & 0.312              &           & 2.33                & 0.36                & 0.10                \\
117.30             & 236.60             & 0.497              &           & 115.88             & 235.77             & 0.494              &           & 1.21                & 0.35                & 0.52                \\
130.80             & 236.10             & 0.556              &           & 129.60             & 235.35             & 0.554              &           & 0.92                & 0.32                & 0.50                \\
                   &                    &                    &           &                    &                    &                    &           &                     &                     &                     \\
150.30             & 236.00             & 0.639              &           & 149.61             & 235.83             & 0.637              &           & 0.46                & 0.07                & 0.20                \\
203.30             & 236.10             & 0.860              &           & 202.19             & 236.21             & 0.861              &           & 0.55                & 0.05               & 0.07               \\
303.50             & 235.30             & 1.291              &           & 302.00             & 235.70             & 1.289              &           & 0.49                & 0.17               & 0.19                \\
493.70             & 235.00             & 2.100              &           & 493.48             & 235.87             & 2.107              &           & 0.04                & 0.37               & 0.35               \\
724.60             & 234.20             & 3.060              &           & 725.43             & 235.56             & 3.100              &           & 0.11               & 0.58               & 1.31               \\
                   &                    &                    &           &                    &                    &                    &           &                     &                     &                     \\
395.20             & 235.60             & 4.132              &           & 420.93             & 236.13             & 4.114              &           & 6.51               & 0.22               & 0.43                \\
404.10             & 235.90             & 4.189              &           & 437.88             & 235.64             & 4.265              &           & 8.36               & 0.11                & 1.80               \\
1384.80            & 233.00             & 6.935              &           & 1409.02            & 234.96             & 6.930              &           & 1.75               & 0.84               & 0.06                \\
1595.40            & 232.00             & 7.057              &           & 1609.30            & 234.95             & 7.022              &           & 0.87               & 1.27               & 0.49                \\
1885.00            & 231.10             & 8.184              &           & 1899.96            & 234.40             & 8.162              &           & 0.79               & 1.43               & 0.27               
\end{tabular}
\caption{Belastungsmessung aufgelistet nach Messbereich}
\label{tab:BelasteteMessung}
\end{table}

\begin{table}[!htbp]
\centering
\begin{tabular}{lll}
\textbf{Multimeter V} & \textbf{Messung V} & \textbf{Abweichung \%}\\ \hline
100.5        & 102.75    & 2.2       \\
150.0        & 151.09    & 0.7       \\
199.5        & 199.88    & 0.2       \\
220.2        & 220.45    & 0.1       \\
230.5        & 230.77    & 0.1       \\
240.1        & 240.36    & 0.1       \\
           
\end{tabular}
\caption{Validierung der Spannungsmessung}
\label{tab:SpannungsmessungTabelle}
\end{table}